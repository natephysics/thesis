% !TEX root = ../my-thesis.tex
%
\pdfbookmark[0]{Abstract}{Abstract}
\addchap*{Abstract}
\label{sec:abstract}

The Wendelstein 7-X (W7-X) plasma experiment, which is the most advanced stellarator of the HELIAS type. The aim of the W7-X is to demonstrate the feasibility of steady-state operation of a plasma experiment and plasma experiment. The W7-X is a five-fold device with a unique magnetic field geometry created using non-planar and planar superconducting coils. It also uses the island-divertor concept for managing heat and particle exhaust, which involves using the edge magnetic field to create plasma-wall interfaces. The graphite divertor targets used in the W7-X are designed to withstand a heat load of up to 10 $MW/m^2$, but exceeding this limit can damage the divertor structures and prevent sustained operation of the device. In order to prevent thermal overload, the magnetic field can be adjusted using trim and control coils. This thesis presents an approach for inferring the edge rotational transform, a key parameter that determines the position of the magnetic islands and heat load pattern, from infrared camera data. Using an inceptnet convolutional neural network, the 520x130 input resolution is a good compromise between computational cost and network performance. This network is able to achieve an $rmse$ of $4.13 \cdot 10^{-3}$ and a training time of just less than a day on a single GPU, which is an order of magnitude better than prior work with similar data.