% !TEX root = ../my-thesis.tex
%
\chapter{Conclusion}
\label{sec:conclusion}

In conclusion, this thesis presents a successful approach for inferring the edge rotational transform, $\bar{\iota}$, of the Wendelstein 7-X plasma experiment from heat load data. By training an inceptnet convolutional neural network on a dataset of infrared camera images, the network was able to infer the edge rotational transform with an $rmse$ of $4.13 \cdot 10^{-3}$. This is an improvement over prior work with similar data and indicates that the proposed approach is a viable solution for determining real-time extraction of $\bar{\iota}$ from the W7-X experiment. Since the network was trained on inputs with a 520x130 resolution, which is a good compromise between computational cost and network performance, further improvements . The training time for the network was less than a day on a single GPU, making it a practical solution for use in the W7-X. Overall, this work demonstrates the potential of using machine learning techniques to improve the control and performance of plasma experiments such as the W7-X.