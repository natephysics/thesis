% !TEX root = ../my-thesis.tex
%
\chapter{Introduction}
\label{sec:intro}

\section{Introduction}
Wendelstein 7-X (W7-X) is the most advanced stellarator of the HELIAS type. Its aim is to demonstrate the feasibility of steady-state operation of a plasma experiment and fusion reactor. It is a five-fold device with a quasi-isodynamic magnetic field created by 50 non-planar and 20 planar superconducting coils. Like its predecessor, Wendelstein 7-AS, it uses the island-divertor concept for heat and particle exhaust. Due to the five-fold symmetry, 10 identical divertor units are arranged so that the divertor targets intersect the magnetic islands present in the edge magnetic field of W7-X and serve as a heat resistent plasma-wall interface. The graphite divertor targets are designed to withstand a heat load of up to $10 \frac{MW}{m^2}$. Exceeding the heat load limit could result in severe damage to the divertor structure and prevent sustained operation of the device.

The dynamics of the particles in the plasma is determined by the magnetic field geometry. The quasi-isodynamic magnetic field geometry in W7-X is characterized by an edge rotational transform $\bar{\iota}$, that is the ratio of polodial turns per toroidal turn of a magnetic field line, which is close to 1 and results in five edge magnetic islands in the so-called standard magnetic field configuration. As the magnetic islands are used to create a plasma-wall interface in the island divertor, the heat and particle deposition pattern strongly depends on the position of the magnetic islands and therefore on the edge rotational transform.

In order to avoid thermal overload of individual divertor targets, the magnetic field can be corrected using a set of trim and control magnetic coils. Real time control of the head load deposition pattern could be used to prevent the thermal overload. In order to provide real-time control control of the state of the plasmas variety of different plasma parameters would need to be provided in real-time, such as $\bar{\iota}$. However, $\bar{\iota}$ cannot be measured directly and simulating $\bar{\iota}$ is computationally expensive and cannot be done in real time. Inferring $\bar{\iota}$ from the heat load pattern is valuable to any such control scheme. In this thesis an approach to inferring $\bar{\iota}$ from the heat load pattern provided by the infrared camera is presented.

Traditionally, to provide an estimate for $\bar{\iota}$ that accounts for the plasmas contribution to the magnetic field, one would need to simulate using a framework like VMEC. The current computational costs makes real time simulation impossible and therefore traditional approaches are not suitable for real time control. In this thesis a convolutional neural network (CNN) is instead used to infer $\bar{\iota}$ from the heat load pattern. Unlike simulations, trained neural nets can be run in real-time. The CNN is trained on a dataset of simulated $\bar{\iota}$ based on real heat load patterns. The CNN is then used to infer $\bar{\iota}$ from the heat load pattern provided by the infrared camera.

One of the questions being addressed by this thesis will be the impact of scaling the input image resolution has on the accuracy of $\bar{\iota}$. To resolve this question, the network is repeatedly trained on a dataset with different input image resolutions and the error in the regression on $\bar{\iota}$ is measured. The distributions of the errors are then compared to help validate the results.